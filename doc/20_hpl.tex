\section{Linpack}
\subsection{One-node performance}
Observant la taula \ref{tab:hpl_one_global_perf}, s'extreu que les configuracions d'NBs i process grid per cada nombre de processos que donen el màxim de GFlops són:
\begin{itemize}
    \item \textbf{1 rank: } Màx =  6.43 GFlops  \\ P = 1  ; Q = 1  ; NB = 256
    \item \textbf{2 ranks:} Màx = 12.22 GFlops  \\ P = 1  ; Q = 2  ; NB = 128
    \item \textbf{4 ranks:} Màx = 23.00 GFlops  \\ P = 2  ; Q = 2  ; NB = 128
    \item \textbf{6 ranks:} Màx = 32.87 GFlops  \\ P = 3  ; Q = 2  ; NB = 128
    \item \textbf{8 ranks:} Màx = 42.58 GFlops  \\P = 4  ; Q = 2  ; NB = 64

\end{itemize}
% taula resultats, resum, posar completa en annex?
% millor nb, process grid

%\begin{table}[]
%    \centering
%\begin{tabular}{cccccc}
    \begin{longtable}{cccccc}
Processes            & P                                           & Q                                           & NB                          & {\color[HTML]{000000} Avg time (s)}                  & {\color[HTML]{000000} Avg GFlops}                    \\ \hline \hline
                     & \cellcolor[HTML]{EFEFEF}                    & \cellcolor[HTML]{EFEFEF}                    & \cellcolor[HTML]{EFEFEF}32  & \cellcolor[HTML]{EFEFEF}{\color[HTML]{000000} 66.46} & \cellcolor[HTML]{EFEFEF}{\color[HTML]{000000} 5.14}  \\
                     & \cellcolor[HTML]{EFEFEF}                    & \cellcolor[HTML]{EFEFEF}                    & 64                          & {\color[HTML]{000000} 56.45}                         & {\color[HTML]{000000} 6.05}                          \\
                     & \cellcolor[HTML]{EFEFEF}                    & \cellcolor[HTML]{EFEFEF}                    & \cellcolor[HTML]{EFEFEF}128 & \cellcolor[HTML]{EFEFEF}{\color[HTML]{000000} 53.86} & \cellcolor[HTML]{EFEFEF}{\color[HTML]{000000} 6.34}  \\
\multirow{-4}{*}{1}  & \multirow{-4}{*}{\cellcolor[HTML]{EFEFEF}1} & \multirow{-4}{*}{\cellcolor[HTML]{EFEFEF}1} & 256                         & {\color[HTML]{000000} 53.14}                         & {\color[HTML]{000000} 6.43}                          \\ \hline
                     &                                             &                                             & \cellcolor[HTML]{EFEFEF}32  & \cellcolor[HTML]{EFEFEF}{\color[HTML]{000000} 32.88} & \cellcolor[HTML]{EFEFEF}{\color[HTML]{000000} 10.38} \\
                     &                                             &                                             & 64                          & {\color[HTML]{000000} 28.74}                         & {\color[HTML]{000000} 11.88}                         \\
                     &                                             &                                             & \cellcolor[HTML]{EFEFEF}128 & \cellcolor[HTML]{EFEFEF}{\color[HTML]{000000} 27.93} & \cellcolor[HTML]{EFEFEF}{\color[HTML]{000000} 12.22} \\
                     & \multirow{-4}{*}{1}                         & \multirow{-4}{*}{2}                         & 256                         & {\color[HTML]{000000} 28.29}                         & {\color[HTML]{000000} 12.07}                         \\ \cline{4-6}
                     & \cellcolor[HTML]{EFEFEF}                    & \cellcolor[HTML]{EFEFEF}                    & \cellcolor[HTML]{EFEFEF}32  & \cellcolor[HTML]{EFEFEF}{\color[HTML]{000000} 33.61} & \cellcolor[HTML]{EFEFEF}{\color[HTML]{000000} 10.16} \\
                     & \cellcolor[HTML]{EFEFEF}                    & \cellcolor[HTML]{EFEFEF}                    & 64                          & {\color[HTML]{000000} 29.67}                         & {\color[HTML]{000000} 11.51}                         \\
                     & \cellcolor[HTML]{EFEFEF}                    & \cellcolor[HTML]{EFEFEF}                    & \cellcolor[HTML]{EFEFEF}128 & \cellcolor[HTML]{EFEFEF}{\color[HTML]{000000} 28.52} & \cellcolor[HTML]{EFEFEF}{\color[HTML]{000000} 11.97} \\
\multirow{-8}{*}{2}  & \multirow{-4}{*}{\cellcolor[HTML]{EFEFEF}2} & \multirow{-4}{*}{\cellcolor[HTML]{EFEFEF}1} & 256                         & {\color[HTML]{000000} 28.88}                         & {\color[HTML]{000000} 11.82}                         \\ \hline
                     &                                             &                                             & \cellcolor[HTML]{EFEFEF}32  & \cellcolor[HTML]{EFEFEF}{\color[HTML]{000000} 17.14} & \cellcolor[HTML]{EFEFEF}{\color[HTML]{000000} 19.92} \\
                     &                                             &                                             & 64                          & {\color[HTML]{000000} 15.26}                         & {\color[HTML]{000000} 22.37}                         \\
                     &                                             &                                             & \cellcolor[HTML]{EFEFEF}128 & \cellcolor[HTML]{EFEFEF}{\color[HTML]{000000} 15.08} & \cellcolor[HTML]{EFEFEF}{\color[HTML]{000000} 22.64} \\
                     & \multirow{-4}{*}{1}                         & \multirow{-4}{*}{4}                         & 256                         & {\color[HTML]{000000} 15.96}                         & {\color[HTML]{000000} 21.39}                         \\ \cline{4-6}
                     & \cellcolor[HTML]{EFEFEF}                    & \cellcolor[HTML]{EFEFEF}                    & \cellcolor[HTML]{EFEFEF}32  & \cellcolor[HTML]{EFEFEF}{\color[HTML]{000000} 17.23} & \cellcolor[HTML]{EFEFEF}{\color[HTML]{000000} 19.81} \\
                     & \cellcolor[HTML]{EFEFEF}                    & \cellcolor[HTML]{EFEFEF}                    & 64                          & {\color[HTML]{000000} 15.19}                         & {\color[HTML]{000000} 22.48}                         \\
                     & \cellcolor[HTML]{EFEFEF}                    & \cellcolor[HTML]{EFEFEF}                    & \cellcolor[HTML]{EFEFEF}128 & \cellcolor[HTML]{EFEFEF}{\color[HTML]{000000} 14.85} & \cellcolor[HTML]{EFEFEF}{\color[HTML]{000000} 23.00} \\
                     & \multirow{-4}{*}{\cellcolor[HTML]{EFEFEF}2} & \multirow{-4}{*}{\cellcolor[HTML]{EFEFEF}2} & 256                         & {\color[HTML]{000000} 15.46}                         & {\color[HTML]{000000} 22.09}                         \\ \cline{4-6}
                     &                                             &                                             & \cellcolor[HTML]{EFEFEF}32  & \cellcolor[HTML]{EFEFEF}{\color[HTML]{000000} 17.82} & \cellcolor[HTML]{EFEFEF}{\color[HTML]{000000} 19.17} \\
                     &                                             &                                             & 64                          & {\color[HTML]{000000} 15.62}                         & {\color[HTML]{000000} 21.87}                         \\
                     &                                             &                                             & \cellcolor[HTML]{EFEFEF}128 & \cellcolor[HTML]{EFEFEF}{\color[HTML]{000000} 15.46} & \cellcolor[HTML]{EFEFEF}{\color[HTML]{000000} 22.08} \\
\multirow{-12}{*}{4} & \multirow{-4}{*}{4}                         & \multirow{-4}{*}{1}                         & 256                         & {\color[HTML]{000000} 16.36}                         & {\color[HTML]{000000} 20.87}                         \\ \hline \pagebreak
                     & \cellcolor[HTML]{EFEFEF}                    & \cellcolor[HTML]{EFEFEF}                    & \cellcolor[HTML]{EFEFEF}32  & \cellcolor[HTML]{EFEFEF}{\color[HTML]{000000} 11.81} & \cellcolor[HTML]{EFEFEF}{\color[HTML]{000000} 28.90} \\
                     & \cellcolor[HTML]{EFEFEF}                    & \cellcolor[HTML]{EFEFEF}                    & 64                          & {\color[HTML]{000000} 10.78}                         & {\color[HTML]{000000} 31.68}                         \\
                     & \cellcolor[HTML]{EFEFEF}                    & \cellcolor[HTML]{EFEFEF}                    & \cellcolor[HTML]{EFEFEF}128 & \cellcolor[HTML]{EFEFEF}{\color[HTML]{000000} 10.76} & \cellcolor[HTML]{EFEFEF}{\color[HTML]{000000} 31.72} \\
                     & \multirow{-4}{*}{\cellcolor[HTML]{EFEFEF}1} & \multirow{-4}{*}{\cellcolor[HTML]{EFEFEF}6} & 256                         & {\color[HTML]{000000} 11.74}                         & {\color[HTML]{000000} 29.08}                         \\ \cline{4-6}
                     &                                             &                                             & \cellcolor[HTML]{EFEFEF}32  & \cellcolor[HTML]{EFEFEF}{\color[HTML]{000000} 11.71} & \cellcolor[HTML]{EFEFEF}{\color[HTML]{000000} 29.15} \\
                     &                                             &                                             & 64                          & {\color[HTML]{000000} 10.46}                         & {\color[HTML]{000000} 32.65}                         \\
                     &                                             &                                             & \cellcolor[HTML]{EFEFEF}128 & \cellcolor[HTML]{EFEFEF}{\color[HTML]{000000} 10.45} & \cellcolor[HTML]{EFEFEF}{\color[HTML]{000000} 32.68} \\
                     & \multirow{-4}{*}{2}                         & \multirow{-4}{*}{3}                         & 256                         & {\color[HTML]{000000} 11.07}                         & {\color[HTML]{000000} 30.84}                         \\ \cline{4-6}
                     & \cellcolor[HTML]{EFEFEF}                    & \cellcolor[HTML]{EFEFEF}                    & \cellcolor[HTML]{EFEFEF}32  & \cellcolor[HTML]{EFEFEF}{\color[HTML]{000000} 11.81} & \cellcolor[HTML]{EFEFEF}{\color[HTML]{000000} 28.92} \\
                     & \cellcolor[HTML]{EFEFEF}                    & \cellcolor[HTML]{EFEFEF}                    & 64                          & {\color[HTML]{000000} 10.52}                         & {\color[HTML]{000000} 32.45}                         \\
                     & \cellcolor[HTML]{EFEFEF}                    & \cellcolor[HTML]{EFEFEF}                    & \cellcolor[HTML]{EFEFEF}128 & \cellcolor[HTML]{EFEFEF}{\color[HTML]{000000} 10.39} & \cellcolor[HTML]{EFEFEF}{\color[HTML]{000000} 32.87} \\
                     & \multirow{-4}{*}{\cellcolor[HTML]{EFEFEF}3} & \multirow{-4}{*}{\cellcolor[HTML]{EFEFEF}2} & 256                         & {\color[HTML]{000000} 11.09}                         & {\color[HTML]{000000} 30.78}                         \\ \cline{4-6}
                     &                                             &                                             & \cellcolor[HTML]{EFEFEF}32  & \cellcolor[HTML]{EFEFEF}{\color[HTML]{000000} 12.49} & \cellcolor[HTML]{EFEFEF}{\color[HTML]{000000} 27.34} \\
                     &                                             &                                             & 64                          & {\color[HTML]{000000} 11.12}                         & {\color[HTML]{000000} 30.69}                         \\
                     &                                             &                                             & \cellcolor[HTML]{EFEFEF}128 & \cellcolor[HTML]{EFEFEF}{\color[HTML]{000000} 11.21} & \cellcolor[HTML]{EFEFEF}{\color[HTML]{000000} 30.45} \\
\multirow{-16}{*}{6} & \multirow{-4}{*}{6}                         & \multirow{-4}{*}{1}                         & 256                         & {\color[HTML]{000000} 12.23}                         & {\color[HTML]{000000} 27.91}                         \\ \hline
                     & \cellcolor[HTML]{EFEFEF}                    & \cellcolor[HTML]{EFEFEF}                    & \cellcolor[HTML]{EFEFEF}32  & \cellcolor[HTML]{EFEFEF}{\color[HTML]{000000} 9.09}  & \cellcolor[HTML]{EFEFEF}{\color[HTML]{000000} 37.54} \\
                     & \cellcolor[HTML]{EFEFEF}                    & \cellcolor[HTML]{EFEFEF}                    & 64                          & {\color[HTML]{000000} 8.35}                          & {\color[HTML]{000000} 40.91}                         \\
                     & \cellcolor[HTML]{EFEFEF}                    & \cellcolor[HTML]{EFEFEF}                    & \cellcolor[HTML]{EFEFEF}128 & \cellcolor[HTML]{EFEFEF}{\color[HTML]{000000} 8.62}  & \cellcolor[HTML]{EFEFEF}{\color[HTML]{000000} 39.63} \\
                     & \multirow{-4}{*}{\cellcolor[HTML]{EFEFEF}1} & \multirow{-4}{*}{\cellcolor[HTML]{EFEFEF}8} & 256                         & {\color[HTML]{000000} 9.73}                          & {\color[HTML]{000000} 35.09}                         \\ \cline{4-6}
                     &                                             &                                             & \cellcolor[HTML]{EFEFEF}32  & \cellcolor[HTML]{EFEFEF}{\color[HTML]{000000} 9.09}  & \cellcolor[HTML]{EFEFEF}{\color[HTML]{000000} 37.56} \\
                     &                                             &                                             & 64                          & {\color[HTML]{000000} 8.06}                          & {\color[HTML]{000000} 42.34}                         \\
                     &                                             &                                             & \cellcolor[HTML]{EFEFEF}128 & \cellcolor[HTML]{EFEFEF}{\color[HTML]{000000} 8.15}  & \cellcolor[HTML]{EFEFEF}{\color[HTML]{000000} 41.88} \\
                     & \multirow{-4}{*}{2}                         & \multirow{-4}{*}{4}                         & 256                         & {\color[HTML]{000000} 8.83}                          & {\color[HTML]{000000} 38.68}                         \\ \cline{4-6}
                     & \cellcolor[HTML]{EFEFEF}                    & \cellcolor[HTML]{EFEFEF}                    & \cellcolor[HTML]{EFEFEF}32  & \cellcolor[HTML]{EFEFEF}{\color[HTML]{000000} 9.12}  & \cellcolor[HTML]{EFEFEF}{\color[HTML]{000000} 37.41} \\
                     & \cellcolor[HTML]{EFEFEF}                    & \cellcolor[HTML]{EFEFEF}                    & 64                          & {\color[HTML]{000000} 8.02}                          & {\color[HTML]{000000} 42.58}                         \\
                     & \cellcolor[HTML]{EFEFEF}                    & \cellcolor[HTML]{EFEFEF}                    & \cellcolor[HTML]{EFEFEF}128 & \cellcolor[HTML]{EFEFEF}{\color[HTML]{000000} 8.11}  & \cellcolor[HTML]{EFEFEF}{\color[HTML]{000000} 42.11} \\
                     & \multirow{-4}{*}{\cellcolor[HTML]{EFEFEF}4} & \multirow{-4}{*}{\cellcolor[HTML]{EFEFEF}2} & 256                         & {\color[HTML]{000000} 8.85}                          & {\color[HTML]{000000} 38.59}                         \\ \cline{4-6}
                     &                                             &                                             & \cellcolor[HTML]{EFEFEF}32  & \cellcolor[HTML]{EFEFEF}{\color[HTML]{000000} 9.91}  & \cellcolor[HTML]{EFEFEF}{\color[HTML]{000000} 34.45} \\
                     &                                             &                                             & 64                          & {\color[HTML]{000000} 8.84}                          & {\color[HTML]{000000} 38.61}                         \\
                     &                                             &                                             & \cellcolor[HTML]{EFEFEF}128 & \cellcolor[HTML]{EFEFEF}{\color[HTML]{000000} 9.10}  & \cellcolor[HTML]{EFEFEF}{\color[HTML]{000000} 37.50} \\
\multirow{-16}{*}{8} & \multirow{-4}{*}{8}                         & \multirow{-4}{*}{1}                         & 256                         & {\color[HTML]{000000} 10.26}                         & {\color[HTML]{000000} 33.27}                         \\ \hline
%\end{tabular}
    \caption{Temps d'execució i performance en un node, per cada MPI rank, grid size (PxQ) i NB.}
    \label{tab:hpl_one_global_perf}
    \end{longtable}
%\end{table}


% plot escalabilitat



\subsection{Two-node performance}

Observant la taula \ref{tab:hpl_two_global_perf}, s'extreu que les configuracions d'NBs i process grid per cada nombre de processos que donen el màxim de GFlops són:
\begin{itemize}
    \item \textbf{2 ranks: } Màx = 12.22 GFlops  \\ P = 1  ; Q = 2  ; NB = 128
    \item \textbf{4 ranks: } Màx = 23.06 GFlops  \\ P = 2  ; Q = 2  ; NB = 128
    \item \textbf{8 ranks: } Màx = 42.65 GFlops  \\ P = 4  ; Q = 2  ; NB = 64
    \item \textbf{12 ranks:} Màx = 60.81 GFlops  \\ P = 4  ; Q = 3  ; NB = 64
    \item \textbf{16 ranks:} Màx = 78.10 GFlops  \\ P = 4  ; Q = 4  ; NB = 64

        Respecte l'execució del benchmark amb 12 processos, cap execució ha pogut computar els resultats més enllà de la combinació de P = 6, Q = 2 i NB = 64. Segurament es degut al límit de temps dels \textit{jobs}.

\end{itemize}
% taula resultats, resum, posar completa en annex?
% millor nb, process grid
% Please add the following required packages to your document preamble:
% \usepackage{multirow}
% \usepackage[table,xcdraw]{xcolor}
% If you use beamer only pass "xcolor=table" option, i.e. \documentclass[xcolor=table]{beamer}
\begin{longtable}{cccccc}
Processes             & P                                                                  & Q                                                                  & NB                                              & {  Avg time (s)}                     & {  Avg GFlops}                       \\ \hline \hline 
                      & \cellcolor[HTML]{EFEFEF}{  }                    & \cellcolor[HTML]{EFEFEF}{  }                    & \cellcolor[HTML]{EFEFEF}32                      & \cellcolor[HTML]{EFEFEF}{  32.95}    & \cellcolor[HTML]{EFEFEF}{  10.36}    \\
                      & \cellcolor[HTML]{EFEFEF}{  }                    & \cellcolor[HTML]{EFEFEF}{  }                    & 64                                              & {  28.75}                            & {  11.88}                            \\
                      & \cellcolor[HTML]{EFEFEF}{  }                    & \cellcolor[HTML]{EFEFEF}{  }                    & \cellcolor[HTML]{EFEFEF}128                     & \cellcolor[HTML]{EFEFEF}{  27.94}    & \cellcolor[HTML]{EFEFEF}{  12.22}    \\
                      & \multirow{-4}{*}{\cellcolor[HTML]{EFEFEF}{  1}} & \multirow{-4}{*}{\cellcolor[HTML]{EFEFEF}{  2}} & 256                                             & {  28.29}                            & {  12.07}                            \\ \cline{2-6}
                      & {  }                                            & {  }                                            & \cellcolor[HTML]{EFEFEF}32                      & \cellcolor[HTML]{EFEFEF}{  33.74}    & \cellcolor[HTML]{EFEFEF}{  10.12}    \\
                      & {  }                                            & {  }                                            & 64                                              & {  29.74}                            & {  11.48}                            \\
                      & {  }                                            & {  }                                            & \cellcolor[HTML]{EFEFEF}128                     & \cellcolor[HTML]{EFEFEF}{  28.52}    & \cellcolor[HTML]{EFEFEF}{  11.97}    \\
\multirow{-8}{*}{2}   & \multirow{-4}{*}{{  2}}                         & \multirow{-4}{*}{{  1}}                         & 256                                             & {  28.88}                            & {  11.82}                            \\ \hline
                      & \cellcolor[HTML]{EFEFEF}{  }                    & \cellcolor[HTML]{EFEFEF}{  }                    & \cellcolor[HTML]{EFEFEF}32                      & \cellcolor[HTML]{EFEFEF}{  17.21}    & \cellcolor[HTML]{EFEFEF}{  19.84}    \\
                      & \cellcolor[HTML]{EFEFEF}{  }                    & \cellcolor[HTML]{EFEFEF}{  }                    & 64                                              & {  15.20}                            & {  22.47}                            \\
                      & \cellcolor[HTML]{EFEFEF}{  }                    & \cellcolor[HTML]{EFEFEF}{  }                    & \cellcolor[HTML]{EFEFEF}128                     & \cellcolor[HTML]{EFEFEF}{  15.07}    & \cellcolor[HTML]{EFEFEF}{  22.66}    \\
                      & \multirow{-4}{*}{\cellcolor[HTML]{EFEFEF}{  1}} & \multirow{-4}{*}{\cellcolor[HTML]{EFEFEF}{  4}} & 256                                             & {  15.95}                            & {  21.40}                            \\ \cline{2-6}
                      & {  }                                            & {  }                                            & \cellcolor[HTML]{EFEFEF}32                      & \cellcolor[HTML]{EFEFEF}{  17.29}    & \cellcolor[HTML]{EFEFEF}{  19.74}    \\
                      & {  }                                            & {  }                                            & 64                                              & {  15.15}                            & {  22.54}                            \\
                      & {  }                                            & {  }                                            & \cellcolor[HTML]{EFEFEF}128                     & \cellcolor[HTML]{EFEFEF}{  14.80}    & \cellcolor[HTML]{EFEFEF}{  23.06}    \\
                      & \multirow{-4}{*}{{  2}}                         & \multirow{-4}{*}{{  2}}                         & 256                                             & {  15.42}                            & {  22.15}                            \\ \cline{2-6}
                      & \cellcolor[HTML]{EFEFEF}{  }                    & \cellcolor[HTML]{EFEFEF}{  }                    & \cellcolor[HTML]{EFEFEF}32                      & \cellcolor[HTML]{EFEFEF}{  17.99}    & \cellcolor[HTML]{EFEFEF}{  18.98}    \\
                      & \cellcolor[HTML]{EFEFEF}{  }                    & \cellcolor[HTML]{EFEFEF}{  }                    & 64                                              & {  15.60}                            & {  21.89}                            \\
                      & \cellcolor[HTML]{EFEFEF}{  }                    & \cellcolor[HTML]{EFEFEF}{  }                    & \cellcolor[HTML]{EFEFEF}128                     & \cellcolor[HTML]{EFEFEF}{  15.45}    & \cellcolor[HTML]{EFEFEF}{  22.10}    \\
\multirow{-12}{*}{4}  & \multirow{-4}{*}{\cellcolor[HTML]{EFEFEF}{  4}} & \multirow{-4}{*}{\cellcolor[HTML]{EFEFEF}{  1}} & 256                                             & {  16.36}                            & {  20.87}                            \\ \hline
                      & {  }                                            & {  }                                            & \cellcolor[HTML]{EFEFEF}32                      & \cellcolor[HTML]{EFEFEF}{  9.07}     & \cellcolor[HTML]{EFEFEF}{  37.63}    \\
                      & {  }                                            & {  }                                            & 64                                              & {  8.33}                             & {  41.00}                            \\
                      & {  }                                            & {  }                                            & \cellcolor[HTML]{EFEFEF}128                     & \cellcolor[HTML]{EFEFEF}{  8.62}     & \cellcolor[HTML]{EFEFEF}{  39.63}    \\
                      & \multirow{-4}{*}{{  1}}                         & \multirow{-4}{*}{{  8}}                         & 256                                             & {  9.72}                             & {  35.13}                            \\ \cline{2-6}
                      & \cellcolor[HTML]{EFEFEF}{  }                    & \cellcolor[HTML]{EFEFEF}{  }                    & \cellcolor[HTML]{EFEFEF}32                      & \cellcolor[HTML]{EFEFEF}{  9.02}     & \cellcolor[HTML]{EFEFEF}{  37.86}    \\
                      & \cellcolor[HTML]{EFEFEF}{  }                    & \cellcolor[HTML]{EFEFEF}{  }                    & 64                                              & {  8.05}                             & {  42.45}                            \\
                      & \cellcolor[HTML]{EFEFEF}{  }                    & \cellcolor[HTML]{EFEFEF}{  }                    & \cellcolor[HTML]{EFEFEF}128                     & \cellcolor[HTML]{EFEFEF}{  8.13}     & \cellcolor[HTML]{EFEFEF}{  41.99}    \\
                      & \multirow{-4}{*}{\cellcolor[HTML]{EFEFEF}{  2}} & \multirow{-4}{*}{\cellcolor[HTML]{EFEFEF}{  4}} & 256                                             & {  8.83}                             & {  38.68}                            \\ \cline{2-6}
                      & {  }                                            & {  }                                            & \cellcolor[HTML]{EFEFEF}32                      & \cellcolor[HTML]{EFEFEF}{  9.06}     & \cellcolor[HTML]{EFEFEF}{  37.66}    \\
                      & {  }                                            & {  }                                            & 64                                              & {  8.01}                             & {  42.65}                            \\
                      & {  }                                            & {  }                                            & \cellcolor[HTML]{EFEFEF}128                     & \cellcolor[HTML]{EFEFEF}{  8.10}     & \cellcolor[HTML]{EFEFEF}{  42.13}    \\
                      & \multirow{-4}{*}{{  4}}                         & \multirow{-4}{*}{{  2}}                         & 256                                             & {  8.85}                             & {  38.59}                            \\ \cline{2-6}
                      & \cellcolor[HTML]{EFEFEF}{  }                    & \cellcolor[HTML]{EFEFEF}{  }                    & \cellcolor[HTML]{EFEFEF}32                      & \cellcolor[HTML]{EFEFEF}{  9.87}     & \cellcolor[HTML]{EFEFEF}{  34.58}    \\
                      & \cellcolor[HTML]{EFEFEF}{  }                    & \cellcolor[HTML]{EFEFEF}{  }                    & 64                                              & {  8.82}                             & {  38.71}                            \\
                      & \cellcolor[HTML]{EFEFEF}{  }                    & \cellcolor[HTML]{EFEFEF}{  }                    & \cellcolor[HTML]{EFEFEF}128                     & \cellcolor[HTML]{EFEFEF}{  9.08}     & \cellcolor[HTML]{EFEFEF}{  37.58}    \\
\multirow{-16}{*}{8}  & \multirow{-4}{*}{\cellcolor[HTML]{EFEFEF}{  8}} & \multirow{-4}{*}{\cellcolor[HTML]{EFEFEF}{  1}} & 256                                             & {  10.22}                            & {  33.42}                            \\ \hline
                      & {  }                                            & {  }                                            & \cellcolor[HTML]{EFEFEF}32                      & \cellcolor[HTML]{EFEFEF}{  6.65}     & \cellcolor[HTML]{EFEFEF}{  51.35}    \\
                      & {  }                                            & {  }                                            & 64                                              & {  6.08}                             & {  56.17}                            \\
                      & {  }                                            & {  }                                            & \cellcolor[HTML]{EFEFEF}128                     & \cellcolor[HTML]{EFEFEF}{  6.46}     & \cellcolor[HTML]{EFEFEF}{  52.87}    \\
                      & \multirow{-4}{*}{{  1}}                         & \multirow{-4}{*}{{  12}}                        & 256                                             & {  7.75}                             & {  44.05}                            \\ \cline{2-6}
                      & \cellcolor[HTML]{EFEFEF}{  }                    & \cellcolor[HTML]{EFEFEF}{  }                    & \cellcolor[HTML]{EFEFEF}32                      & \cellcolor[HTML]{EFEFEF}{  6.35}     & \cellcolor[HTML]{EFEFEF}{  53.74}    \\
                      & \cellcolor[HTML]{EFEFEF}{  }                    & \cellcolor[HTML]{EFEFEF}{  }                    & 64                                              & {  5.72}                             & {  59.64}                            \\
                      & \cellcolor[HTML]{EFEFEF}{  }                    & \cellcolor[HTML]{EFEFEF}{  }                    & \cellcolor[HTML]{EFEFEF}128                     & \cellcolor[HTML]{EFEFEF}{  5.85}     & \cellcolor[HTML]{EFEFEF}{  58.37}    \\
                      & \multirow{-4}{*}{\cellcolor[HTML]{EFEFEF}{  2}} & \multirow{-4}{*}{\cellcolor[HTML]{EFEFEF}{  6}} & 256                                             & {  6.53}                             & {  52.31}                            \\ \cline{2-6}
                      & {  }                                            & {  }                                            & \cellcolor[HTML]{EFEFEF}32                      & \cellcolor[HTML]{EFEFEF}{  6.37}     & \cellcolor[HTML]{EFEFEF}{  53.62}    \\
                      & {  }                                            & {  }                                            & 64                                              & {  5.62}                             & {  60.75}                            \\
                      & {  }                                            & {  }                                            & \cellcolor[HTML]{EFEFEF}128                     & \cellcolor[HTML]{EFEFEF}{  5.82}     & \cellcolor[HTML]{EFEFEF}{  58.71}    \\
                      & \multirow{-4}{*}{{  3}}                         & \multirow{-4}{*}{{  4}}                         & 256                                             & {  6.42}                             & {  53.14}                            \\ \cline{2-6}
                      & \cellcolor[HTML]{EFEFEF}{  }                    & \cellcolor[HTML]{EFEFEF}{  }                    & \cellcolor[HTML]{EFEFEF}32                      & \cellcolor[HTML]{EFEFEF}{  6.37}     & \cellcolor[HTML]{EFEFEF}{  53.61}    \\
                      & \cellcolor[HTML]{EFEFEF}{  }                    & \cellcolor[HTML]{EFEFEF}{  }                    & 64                                              & {  5.62}                             & {  60.81}                            \\
                      & \cellcolor[HTML]{EFEFEF}{  }                    & \cellcolor[HTML]{EFEFEF}{  }                    & \cellcolor[HTML]{EFEFEF}128                     & \cellcolor[HTML]{EFEFEF}{  5.73}     & \cellcolor[HTML]{EFEFEF}{  59.61}    \\
                      & \multirow{-4}{*}{\cellcolor[HTML]{EFEFEF}{  4}} & \multirow{-4}{*}{\cellcolor[HTML]{EFEFEF}{  3}} & 256                                             & {  6.37}                             & {  53.60}                            \\ \cline{2-6}
                      & {  }                                            & {  }                                            & \cellcolor[HTML]{EFEFEF}32                      & \cellcolor[HTML]{EFEFEF}{  6.49}     & \cellcolor[HTML]{EFEFEF}{  52.64}    \\
                      & {  }                                            & {  }                                            & 64                                              & {  5.73}                             & {  59.56}                            \\
                      & {  }                                            & {  }                                            & \cellcolor[HTML]{EFEFEF}128                     & \cellcolor[HTML]{EFEFEF}{  -} & \cellcolor[HTML]{EFEFEF}{  -} \\
                      & \multirow{-4}{*}{{  6}}                         & \multirow{-4}{*}{{  2}}                         & 256                                             & {  -}                         & {  -}                         \\ \cline{2-6}
                      & \cellcolor[HTML]{EFEFEF}{  }                    & \cellcolor[HTML]{EFEFEF}{  }                    & \multicolumn{1}{l}{\cellcolor[HTML]{EFEFEF}32}  & \cellcolor[HTML]{EFEFEF}{  6.65}     & \cellcolor[HTML]{EFEFEF}{  51.35}    \\
                      & \cellcolor[HTML]{EFEFEF}{  }                    & \cellcolor[HTML]{EFEFEF}{  }                    & \multicolumn{1}{l}{64}                          & {  6.08}                             & {  56.17}                            \\
                      & \cellcolor[HTML]{EFEFEF}{  }                    & \cellcolor[HTML]{EFEFEF}{  }                    & \multicolumn{1}{l}{\cellcolor[HTML]{EFEFEF}128} & \cellcolor[HTML]{EFEFEF}{  6.46}     & \cellcolor[HTML]{EFEFEF}{  52.87}    \\
\multirow{-24}{*}{12} & \multirow{-4}{*}{\cellcolor[HTML]{EFEFEF}{  1}} & \multirow{-4}{*}{\cellcolor[HTML]{EFEFEF}{  12}}& \multicolumn{1}{l}{256}                         & {  7.75}                             & {  44.05}                            \\ \hline
                      & {  }                                            & {  }                                            & \multicolumn{1}{l}{\cellcolor[HTML]{EFEFEF}32}  & \cellcolor[HTML]{EFEFEF}{  5.31}     & \cellcolor[HTML]{EFEFEF}{  64.37}    \\
                      & {  }                                            & {  }                                            & \multicolumn{1}{l}{64}                          & {  4.98}                             & {  68.51}                            \\
                      & {  }                                            & {  }                                            & \multicolumn{1}{l}{\cellcolor[HTML]{EFEFEF}128} & \cellcolor[HTML]{EFEFEF}{  5.40}     & \cellcolor[HTML]{EFEFEF}{  63.20}    \\
                      & \multirow{-4}{*}{{  1}}                         & \multirow{-4}{*}{{  16}}                        & \multicolumn{1}{l}{256}                         & {  6.54}                             & {  52.22}                            \\ \cline{2-6}
                      & \cellcolor[HTML]{EFEFEF}{  }                    & \cellcolor[HTML]{EFEFEF}{  }                    & \multicolumn{1}{l}{\cellcolor[HTML]{EFEFEF}32}  & \cellcolor[HTML]{EFEFEF}{  4.91}     & \cellcolor[HTML]{EFEFEF}{  69.48}    \\
                      & \cellcolor[HTML]{EFEFEF}{  }                    & \cellcolor[HTML]{EFEFEF}{  }                    & \multicolumn{1}{l}{64}                          & {  4.56}                             & {  74.86}                            \\
                      & \cellcolor[HTML]{EFEFEF}{  }                    & \cellcolor[HTML]{EFEFEF}{  }                    & \multicolumn{1}{l}{\cellcolor[HTML]{EFEFEF}128} & \cellcolor[HTML]{EFEFEF}{  4.72}     & \cellcolor[HTML]{EFEFEF}{  72.41}    \\
                      & \multirow{-4}{*}{\cellcolor[HTML]{EFEFEF}{  2}}  & \multirow{-4}{*}{\cellcolor[HTML]{EFEFEF}{  8}} & \multicolumn{1}{l}{256}                         & {  5.45}                             & {  62.68}                            \\ \cline{2-6}
                      & {  }                                            & {  }                                            & \multicolumn{1}{l}{\cellcolor[HTML]{EFEFEF}32}  & \cellcolor[HTML]{EFEFEF}{  4.94}     & \cellcolor[HTML]{EFEFEF}{  69.21}    \\
                      & {  }                                            & {  }                                            & \multicolumn{1}{l}{64}                          & {  4.37}                             & {  78.10}                            \\
                      & {  }                                            & {  }                                            & \multicolumn{1}{l}{\cellcolor[HTML]{EFEFEF}128} & \cellcolor[HTML]{EFEFEF}{  4.52}     & \cellcolor[HTML]{EFEFEF}{  75.49}    \\
                      & \multirow{-4}{*}{{  4}}                         & \multirow{-4}{*}{{  4}}                         & \multicolumn{1}{l}{256}                         & {  5.14}                             & {  66.49}                            \\ \cline{2-6}
                      & \cellcolor[HTML]{EFEFEF}{  }                    & \cellcolor[HTML]{EFEFEF}{  }                    & \multicolumn{1}{l}{\cellcolor[HTML]{EFEFEF}32}  & \cellcolor[HTML]{EFEFEF}{  5.42}     & \cellcolor[HTML]{EFEFEF}{  63.05}    \\
                      & \cellcolor[HTML]{EFEFEF}{  }                    & \cellcolor[HTML]{EFEFEF}{  }                    & \multicolumn{1}{l}{64}                          & {  4.70}                             & {  72.69}                            \\
                      & \cellcolor[HTML]{EFEFEF}{  }                    & \cellcolor[HTML]{EFEFEF}{  }                    & \multicolumn{1}{l}{\cellcolor[HTML]{EFEFEF}128} & \cellcolor[HTML]{EFEFEF}{  4.96}     & \cellcolor[HTML]{EFEFEF}{  68.84}    \\
                      & \multirow{-4}{*}{\cellcolor[HTML]{EFEFEF}{  8}} & \multirow{-4}{*}{\cellcolor[HTML]{EFEFEF}{  2}} & \multicolumn{1}{l}{256}                         & {  5.73}                             & {  59.62}                            \\ \cline{2-6}
                      & {  }                                            & {  }                                            & \multicolumn{1}{l}{\cellcolor[HTML]{EFEFEF}32}  & \cellcolor[HTML]{EFEFEF}{  5.31}     & \cellcolor[HTML]{EFEFEF}{  64.37}    \\
                      & {  }                                            & {  }                                            & \multicolumn{1}{l}{64}                          & {  4.98}                             & {  68.51}                            \\
                      & {  }                                            & {  }                                            & \multicolumn{1}{l}{\cellcolor[HTML]{EFEFEF}128} & \cellcolor[HTML]{EFEFEF}{  5.40}     & \cellcolor[HTML]{EFEFEF}{  63.20}    \\
\multirow{-20}{*}{16} & \multirow{-4}{*}{{  1}}                         & \multirow{-4}{*}{{  16}}                        & \multicolumn{1}{l}{256}                         & {  6.54}                             & {  52.22} \\ \hline
    \caption{Temps d'execució i performance en dos noed, per cada MPI rank, grid size (PxQ) i NB.}
    \label{tab:hpl_two_global_perf}
\end{longtable}

% plot escalabilitat

\subsection{Peak performance}
La \textit{peak performance} teòrica de cada core del processador Intel Xeon E5-2609 v4 és de:

\[2\ units\ *\ (256/64)\ ops\_per\_cicle\ *\ 2\ flops\_per\_cycle\ *\ 1.7\ GHz\ =\ 27.2\ GFlops\]

La \textit{peak performance} teòrica del node, amb 2 processadors Intel Xeon E5-2609 v4 és de:

\[2\ proc\ *\ 8\ cores/proc\ *\ 27.2\ GFlops\_per\_core\ =\ 435.2\ GFlops\]

Comparant la performance obtinguda amb HPL i la teòrica, observem que aquesta primera dista molt de la que hauriem d'aconseguir. Ens centrem en la combinació que millor GFlops ha obtingut, 16 processos en 2 nodes, amb 78.10 GFlops, en comparació amb els 435.2 GFlops teòrics (coincideix amb la performance teòrica del node perquè fa ús de 16 cores). 
S'està obtenint una performance 5.6 cops inferior.

